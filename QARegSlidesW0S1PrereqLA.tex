% Options for packages loaded elsewhere
\PassOptionsToPackage{unicode}{hyperref}
\PassOptionsToPackage{hyphens}{url}
\PassOptionsToPackage{dvipsnames,svgnames,x11names}{xcolor}
%
\documentclass[
  letterpaper,
  DIV=11,
  numbers=noendperiod]{scrartcl}

\usepackage{amsmath,amssymb}
\usepackage{iftex}
\ifPDFTeX
  \usepackage[T1]{fontenc}
  \usepackage[utf8]{inputenc}
  \usepackage{textcomp} % provide euro and other symbols
\else % if luatex or xetex
  \usepackage{unicode-math}
  \defaultfontfeatures{Scale=MatchLowercase}
  \defaultfontfeatures[\rmfamily]{Ligatures=TeX,Scale=1}
\fi
\usepackage{lmodern}
\ifPDFTeX\else  
    % xetex/luatex font selection
\fi
% Use upquote if available, for straight quotes in verbatim environments
\IfFileExists{upquote.sty}{\usepackage{upquote}}{}
\IfFileExists{microtype.sty}{% use microtype if available
  \usepackage[]{microtype}
  \UseMicrotypeSet[protrusion]{basicmath} % disable protrusion for tt fonts
}{}
\makeatletter
\@ifundefined{KOMAClassName}{% if non-KOMA class
  \IfFileExists{parskip.sty}{%
    \usepackage{parskip}
  }{% else
    \setlength{\parindent}{0pt}
    \setlength{\parskip}{6pt plus 2pt minus 1pt}}
}{% if KOMA class
  \KOMAoptions{parskip=half}}
\makeatother
\usepackage{xcolor}
\setlength{\emergencystretch}{3em} % prevent overfull lines
\setcounter{secnumdepth}{-\maxdimen} % remove section numbering
% Make \paragraph and \subparagraph free-standing
\makeatletter
\ifx\paragraph\undefined\else
  \let\oldparagraph\paragraph
  \renewcommand{\paragraph}{
    \@ifstar
      \xxxParagraphStar
      \xxxParagraphNoStar
  }
  \newcommand{\xxxParagraphStar}[1]{\oldparagraph*{#1}\mbox{}}
  \newcommand{\xxxParagraphNoStar}[1]{\oldparagraph{#1}\mbox{}}
\fi
\ifx\subparagraph\undefined\else
  \let\oldsubparagraph\subparagraph
  \renewcommand{\subparagraph}{
    \@ifstar
      \xxxSubParagraphStar
      \xxxSubParagraphNoStar
  }
  \newcommand{\xxxSubParagraphStar}[1]{\oldsubparagraph*{#1}\mbox{}}
  \newcommand{\xxxSubParagraphNoStar}[1]{\oldsubparagraph{#1}\mbox{}}
\fi
\makeatother


\providecommand{\tightlist}{%
  \setlength{\itemsep}{0pt}\setlength{\parskip}{0pt}}\usepackage{longtable,booktabs,array}
\usepackage{calc} % for calculating minipage widths
% Correct order of tables after \paragraph or \subparagraph
\usepackage{etoolbox}
\makeatletter
\patchcmd\longtable{\par}{\if@noskipsec\mbox{}\fi\par}{}{}
\makeatother
% Allow footnotes in longtable head/foot
\IfFileExists{footnotehyper.sty}{\usepackage{footnotehyper}}{\usepackage{footnote}}
\makesavenoteenv{longtable}
\usepackage{graphicx}
\makeatletter
\def\maxwidth{\ifdim\Gin@nat@width>\linewidth\linewidth\else\Gin@nat@width\fi}
\def\maxheight{\ifdim\Gin@nat@height>\textheight\textheight\else\Gin@nat@height\fi}
\makeatother
% Scale images if necessary, so that they will not overflow the page
% margins by default, and it is still possible to overwrite the defaults
% using explicit options in \includegraphics[width, height, ...]{}
\setkeys{Gin}{width=\maxwidth,height=\maxheight,keepaspectratio}
% Set default figure placement to htbp
\makeatletter
\def\fps@figure{htbp}
\makeatother

\usepackage{graphicx} % Allows including images
\usepackage{booktabs}
\usepackage{amsmath}
\usepackage{url}
\usepackage{subfigure}
\usepackage{xcolor}
\usepackage{amsthm,amsfonts,amssymb,amscd,amsxtra}
\usepackage{pgfpages}
\usepackage{animate}
\usepackage[backend=bibtex,style=verbose-trad2]{biblatex}
\bibliography{bibliography}
\usepackage{mathpazo}
\usepackage{unicode-math}
\KOMAoption{captions}{tableheading}
\makeatletter
\@ifpackageloaded{caption}{}{\usepackage{caption}}
\AtBeginDocument{%
\ifdefined\contentsname
  \renewcommand*\contentsname{Table of contents}
\else
  \newcommand\contentsname{Table of contents}
\fi
\ifdefined\listfigurename
  \renewcommand*\listfigurename{List of Figures}
\else
  \newcommand\listfigurename{List of Figures}
\fi
\ifdefined\listtablename
  \renewcommand*\listtablename{List of Tables}
\else
  \newcommand\listtablename{List of Tables}
\fi
\ifdefined\figurename
  \renewcommand*\figurename{Figure}
\else
  \newcommand\figurename{Figure}
\fi
\ifdefined\tablename
  \renewcommand*\tablename{Table}
\else
  \newcommand\tablename{Table}
\fi
}
\@ifpackageloaded{float}{}{\usepackage{float}}
\floatstyle{ruled}
\@ifundefined{c@chapter}{\newfloat{codelisting}{h}{lop}}{\newfloat{codelisting}{h}{lop}[chapter]}
\floatname{codelisting}{Listing}
\newcommand*\listoflistings{\listof{codelisting}{List of Listings}}
\makeatother
\makeatletter
\makeatother
\makeatletter
\@ifpackageloaded{caption}{}{\usepackage{caption}}
\@ifpackageloaded{subcaption}{}{\usepackage{subcaption}}
\makeatother

\ifLuaTeX
  \usepackage{selnolig}  % disable illegal ligatures
\fi
\usepackage{bookmark}

\IfFileExists{xurl.sty}{\usepackage{xurl}}{} % add URL line breaks if available
\urlstyle{same} % disable monospaced font for URLs
\hypersetup{
  pdftitle={Q\&A},
  colorlinks=true,
  linkcolor={blue},
  filecolor={Maroon},
  citecolor={Blue},
  urlcolor={Blue},
  pdfcreator={LaTeX via pandoc}}


\title{Q\&A}
\usepackage{etoolbox}
\makeatletter
\providecommand{\subtitle}[1]{% add subtitle to \maketitle
  \apptocmd{\@title}{\par {\large #1 \par}}{}{}
}
\makeatother
\subtitle{Slide deck 1: Linear algebra and statistical prerequisites}
\author{}
\date{}

\begin{document}
\maketitle


\section{Test}\label{test}

\begin{itemize}
\tightlist
\item
  What is the notation for the i-th row of a matrix \(\symbf{A}\)?

  \begin{itemize}
  \tightlist
  \item
    \(\symbf{a}_{[i]}\)
  \item
    Is \(\symbf{a}_{[i]}\) a row or column vector?

    \begin{itemize}
    \tightlist
    \item
      Column vector
    \end{itemize}
  \end{itemize}
\item
  What is the notation for the i-th column of a matrix \(\symbf{A}\)?

  \begin{itemize}
  \tightlist
  \item
    \(\symbf{a}_{i}\)
  \item
    Is \(\symbf{a}_{i}\) a row or column vector?

    \begin{itemize}
    \tightlist
    \item
      Column vector
    \end{itemize}
  \end{itemize}
\item
  How do you calculate the \(ij\)-th element in matrix multiplication
  \(\symbf{AB}\)?

  \begin{itemize}
  \tightlist
  \item
    Dot product of i-th row of \(\symbf{A}\) and j-th column of
    \(\symbf{B}\)
  \end{itemize}
\item
  What are the conditions for orthogonality between two vectors
  \(\symbf{x}\) and \(\symbf{y}\):

  \begin{itemize}
  \tightlist
  \item
    Algebraic condition?

    \begin{itemize}
    \tightlist
    \item
      Their dot product is zero
    \end{itemize}
  \item
    Geometric condition?

    \begin{itemize}
    \tightlist
    \item
      The angle between the vectors is 90 degrees
    \end{itemize}
  \end{itemize}
\item
  Condition for a matrix being column-wise orthogonal?

  \begin{itemize}
  \tightlist
  \item
    Every column is orthogonal to every other column
  \end{itemize}
\item
  Conditions for a matrix being orthogonal:

  \begin{itemize}
  \tightlist
  \item
    First?

    \begin{itemize}
    \tightlist
    \item
      Square
    \end{itemize}
  \item
    Second?

    \begin{itemize}
    \tightlist
    \item
      Every column is orthogonal to every other column
    \end{itemize}
  \end{itemize}
\item
  Conditions for a matrix being orthonormal:

  \begin{itemize}
  \tightlist
  \item
    First?

    \begin{itemize}
    \tightlist
    \item
      Orthogonal
    \end{itemize}
  \item
    Second?

    \begin{itemize}
    \tightlist
    \item
      Every column has unit length
    \end{itemize}
  \end{itemize}
\item
  How is the trace of a matrix calculated?

  \begin{itemize}
  \tightlist
  \item
    Sum of diagonal elements
  \end{itemize}
\item
  What is the cyclic property of traces?

  \begin{itemize}
  \tightlist
  \item
    \(\mathrm{tr}(\symbf{ABC})= \mathrm{tr}(\symbf{CAB})= \mathrm{tr}(\symbf{BCA})\)
  \end{itemize}
\item
  What is the maximum possible rank of a rectangular matrix?

  \begin{itemize}
  \tightlist
  \item
    Its shorter dimension
  \end{itemize}
\item
  What is an eigenvector \(\symbf{x}\) of a matrix \(\symbf{A}\)?

  \begin{itemize}
  \tightlist
  \item
    A unit-length vector such that
    \(\symbf{A}\symbf{x}=\lambda\symbf{x}\)
  \end{itemize}
\item
  What can any symmetric matrix \(\symbf{S}\) be factorisation as?

  \begin{itemize}
  \tightlist
  \item
    \(\symbf{P}\symbf{\Lambda}\symbf{P}'\)
  \item
    What is:

    \begin{itemize}
    \tightlist
    \item
      \(\symbf{P}\)?

      \begin{itemize}
      \tightlist
      \item
        Matrix of eigenvectors of \(\symbf{S}\)
      \item
        Are the eigenvectors the rows or columns of \(\symbf{P}\)?

        \begin{itemize}
        \tightlist
        \item
          Columns
        \end{itemize}
      \end{itemize}
    \item
      \(\symbf{\Lambda}\)?

      \begin{itemize}
      \tightlist
      \item
        Diagonal matrix of eigenvalues of \(\symbf{S}\)
      \end{itemize}
    \item
      What is a name commonly given to the factorisation
      \(\symbf{S}= \symbf{P}\symbf{\Lambda}\symbf{P}'\)?

      \begin{itemize}
      \tightlist
      \item
        Spectral decomposition
      \end{itemize}
    \end{itemize}
  \end{itemize}
\item
  What characterizes a positive definite matrix?

  \begin{itemize}
  \tightlist
  \item
    All eigenvalues are positive
  \end{itemize}
\item
  What is the condition for a matrix \(\symbf{A}\) to be idempotent?

  \begin{itemize}
  \tightlist
  \item
    \(\symbf{A}^2=\symbf{A}\symbf{A}=\symbf{A}\)
  \end{itemize}
\item
  What are the eigenvalues of an idempotent matrix?

  \begin{itemize}
  \tightlist
  \item
    0 and 1
  \end{itemize}
\item
  What is the factorization of an idempotent matrix?

  \begin{itemize}
  \tightlist
  \item
    \(\symbf{P}\symbf{\Lambda}\symbf{P}'\)
  \item
    What is:

    \begin{itemize}
    \tightlist
    \item
      \(\symbf{\Lambda}\)?

      \begin{itemize}
      \tightlist
      \item
        \(\begin{bmatrix} \symbf{I}_r & 0 \\ 0 & 0 \end{bmatrix}\)
      \end{itemize}
    \item
      \(\symbf{P}\)?

      \begin{itemize}
      \tightlist
      \item
        Matrix of eigenvectors of \(\symbf{A}\)
      \item
        Are the eigenvectors the rows or columns of \(\symbf{P}\)?

        \begin{itemize}
        \tightlist
        \item
          Columns
        \end{itemize}
      \end{itemize}
    \end{itemize}
  \end{itemize}
\end{itemize}




\end{document}
